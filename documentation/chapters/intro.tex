\chapter{Bevezetés} % Introduction
\label{ch:intro}


A játékkonzol-emulátorok feladata, hogy egy kompatibilitási réteget képezzenek a modern x86 és ARM architektúrájú processzorok, valamint az elavult konzolokra megjelent játékok között, hogy azokat bárki zavartalanul élvezhesse a konzol birtoklása nélkül. Az emulációt végző programnak szoftveresen kell megvalósítania az eredeti konzol számítási egységei által nyújtott primitív utasításokat és a komponensek közötti kommunikációt, hogy a játékok az elvárt viselkedés szerint működjenek.

Az emulátorok világában a hardverközeli, elsősorban teljesítményre kihegyezett nyelvek használata (pl. C++) az elterjedt, mivel ezen a területen a program gyorsasága kulcsfontosságú. Ezeknél a nyelveknél az explicit memóriakezelés és a vékony absztrakciós réteg megkönnyíti az optimalizációt, azonban ennek a kód átláthatósága látja a kárát.
Ezzel szemben a funkcionális nyelvek erős kifejezőképessége és moduláris felépítést előnyben részesítő paradigmája az emulátorfejlesztésnél számos helyzetben könnyítik meg a programozó dolgát. A szakdolgozatom célja, hogy korszerű eszközök segítségével
egy fejlesztőbarát, de mégis hatékony emulátort implementáljak funkcionális nyelven. A megfelelő teljesítményről a Haskell nyelv egyeduralkodó fordítója, a GHC gondoskodik, ami az egyik legfejlettebb fordítóprogram, ami funkcionális nyelvhez készült. Agresszív optimalizálási stratégiáján túl az is mellette szól, hogy a legfrissebb kiadása immár tartalmaz egy valós idejű alkalmazásokhoz szánt, alacsony késleltetésű szemétgyűjtőt. 

A Nintendo Entertainment System (NES) generációjának legsikeresebb konzolja, több mint 61 millió darab kelt el belőle világszerte. Emiatt kezdettől fogva nagy volt az igény az emulátorokra és mára a hardver nem publikus részei is fel lettek térképezve. Választásom azért erre a rendszerre esett, mert az internetről elérhető dokumentációk és leírások birtokában nincs szükség a konzol beszerzésére, a hardver viselkedésének felderítésére.